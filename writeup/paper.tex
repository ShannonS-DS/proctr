\documentclass[letterpaper, 10pt, conference]{ieeeconf}


\IEEEoverridecommandlockouts
\overrideIEEEmargins

\let\proof\relax
\let\endproof\relax

\usepackage{amsmath, amssymb}
\usepackage{url}
\usepackage[pdftex]{graphicx}
\usepackage{float}
\usepackage{relsize}
\usepackage{algorithm}
\usepackage{fancyvrb}
\usepackage[noend]{algorithmic}
\usepackage{subfiles}
\usepackage{titlecaps}
\usepackage{fancyhdr}
\usepackage{amsthm}

\renewcommand{\thefootnote}{\fnsymbol{footnote}}
\renewcommand{\algorithmicrequire}{\textbf{Input:}}
\renewcommand{\algorithmicensure}{\textbf{Output:}}

\newcommand{\tab}[1]{\hspace{.2\textwidth}\rlap{#1}}
\newcommand{\itab}[1]{\hspace{0em}\rlap{#1}}
\renewcommand{\thefootnote}{\fnsymbol{footnote}}
\newcommand{\Acronym}[1]{\ensuremath{{{\texttt{#1}}}}}
\newcommand{\Symbol}[1]{\ensuremath{\mathcal{#1}}}
\newcommand{\Function}[1]{\ensuremath{{\textsc{#1}}}}
\newcommand{\Area}[0]{\ensuremath{{\textsc{Area}}}}
\newcommand{\Constant}[1]{\ensuremath{{\texttt{#1}}}}
\newcommand{\Var}[1]{\ensuremath{{{\textsl{#1}}}}}
\newcommand{\rB}[0]{\ensuremath{{{\mathcal{B'}}}}}
\newcommand{\B}[0]{\ensuremath{{{\mathcal{B}}}}}
\newcommand{\False}{\Constant{false}}
\newcommand{\True}{\Constant{true}}
\newcommand{\Null}{\Constant{null}}
\newcommand{\dif}{\ensuremath{{\mathrm{d}}}}
\newcommand{\where}{\ensuremath{{\textbf{ where }}}}
\newcommand{\Name}{\Acronym{Dodger}}
\newcommand{\Revision}[1]{\textcolor{red}{#1}}
\newcommand{\R}{\ensuremath{\mathbb{R}}}
\newcommand{\Traj}{\ensuremath{\zeta}}
\newcommand{\Tree}{\Symbol{T}}
\newcommand{\pair}[1]{\ensuremath{\langle#1\rangle}}
\newcommand{\argmin}[1]{\underset{#1}{\operatorname{arg}\,\operatorname{min}}\;}
\newcommand{\argmax}[1]{\underset{#1}{\operatorname{arg}\,\operatorname{max}}\;}
\newcommand{\Max}[1]{\underset{#1}{\operatorname{max}}\;}
\newcommand{\mychange}[1]{\textcolor{red}{#1}}
\newcommand{\Sum}[1]{\displaystyle\sum\limits_{#1}}
\newcommand{\Int}[1]{\displaystyle\int\limits_{#1}}
\newcommand{\bx}[0]{\ensuremath{\mathbf{x}}}
\newcommand{\by}[0]{\ensuremath{\mathbf{y}}}

\newtheorem{theorem}{Theorem}
\newtheorem{lemma}{Lemma}
\newtheorem{corollary}{Corollary}
\newtheorem{definition}{Definition}

\begin{document}

\title{Vehicle Rebalancing for Ridesharing Enabled Mobility On Demand Systems}

\author{Alex Wallar}

\maketitle

% \begin{abstract}
%
% \end{abstract}

\section{Introduction}

Enabling ridesharing for mobility on demand (MOD) systems requires passengers
to specify a maximum waiting time, the maximum amount of time they are willing
to wait for a vehicle to pick them up. This constraint allows us to develop
shareability networks which in turn helps us to decide which passengers can be
shared with which rides. The unfortunate consequence of this approach is that
it can leave vehicles underutilized and passengers without rides. This
fundamental issue can be addressed by intelligently rebalancing vehicles in the
road network to ensure that the maximum amount of rides can be serviced.  Our
approach will have two stages. Firstly, vehicles which have already accepted
passengers will be routed through areas that have a high probability of
requests that could also be shared with the current passenger demands whilst
keeping the path within the time constraints given by the passengers.  We will
call this \emph{active rebalancing}. Secondly, any vehicles that cannot be
assigned to passengers will move to regions that have a high probability of any
request without overburdening the congestion limits of the road network. We
will call this \emph{passive rebalancing}. These two problems are solved in
sequence by first actively rebalancing the currently occupied vehicles, and
then using the routes they have chosen to perform passive rebalancing. Both
rebalancing schemes are formulated as constrained optimization problems with
integer constraints. Particularly, active and passive rebalancing are
formulated as a mixed integer linear program (MILP) and a mixed integer
quadratically constrained program (MIQCP).

In order to perform active and passive rebalancing, we need to be able to infer
the time varying rate at which a particular route is being requested and the
rate a certain region is producing requests. We use a particle filter to
estimate these rates in real-time and use historical taxi request data to learn
a time varying multimodal prior distribution to aid in inferring noisy rates.

\section{Problem Formulation}

The road network for the MOD system is modelled as a directed graph denoted by
$\mathcal{G} = (\mathcal{N}, \mathcal{E})$ where $\mathcal{N} = \{n_0, \ldots,
n_{N_n}\} \subset \mathbb{R}^2$ is the set of nodes and $\mathcal{E} = \{e_0,
\ldots, e_{N_e}\}$ is the set of directed edges represented as ordered pairs,
$\forall i,\, e_i \in \mathcal{N}^2$. Let the set of incoming requests for
rides at time $t$ be denoted as $\mathcal{R}_{t} \subseteq \mathcal{E}$ and let
$\Pi(\mathcal{R}_t) = \{\pi_0, \ldots, \pi_{N_v}\}$ be the set of optimal paths
given to each vehicle to satisfy $\mathcal{R}_t$ given by the ridsharing
algorithm in \cite{alonsomora16pnas}. Each $\pi_i \in \Pi(\mathcal{R}_t)$
denotes the path vehicle $i$ would be assigned to contribute to satisfying the
requests in $\mathcal{R}_t$ with $\pi_i \in 2^{\mathcal{N}}$ and $(\pi^k_i,
\pi^{k + 1}_i) \in \mathcal{E}$. Note that $\pi_i$ may be the empty set. Let
$||\pi_i||$ represent the total travel time of the path.

\subsection{Estimating Request Rates}

The operating area for the MOD system is partitioned into $N_{k}$ regions
denoted as $\mathcal{K} = \{k_0, \ldots, k_{N_k}\}$. A \emph{route} is defined
as origin-destination pair represented as an ordered pair, $\rho \in
\mathcal{K}^2$. Rides are assumed to be requested according to an inhomogeneous
Poisson point process with a stochastic time-varying rate, $\hat{\lambda}(t)$.
These rates are assumed to drift over time in a short horizon according to a
Wiener process.  This means $\hat\lambda(t') - \hat\lambda(t)
\sim \mathfrak{N}(0, \sigma^2 \cdot (t' - t))$, where $\mathfrak{N}$ is a
normal distribution and $\sigma^2$ represents the volatility of the process. We
also assume we have access to historical taxi request data in order to build a
prior distribution.  We seek to determine the rate that different routes are
being requested and the rate that regions are producing requests in the
presence of noise and uncertainty.

\subsection{Active Rebalancing}

As with the work in \cite{alonsomora16pnas} we assume the there is a maximum
acceptable time a potential passenger will spend waiting before being picked up
denoted as $w$. Now for each vehicle, $v$, we seek to find a Pareto optimal
path, $\pi_v$, that minimizes the added travel time, $||\pi_v|| - ||\pi_v^*||$,
and maximizes the added probability of filling the vehicle to capacity, where
$\pi_v^* \in \Pi(\mathcal{R}_t)$ is the original optimal path for the vehicle.

\subsection{Passive Rebalancing}

For passive rebalancing, we seek to direct cars to regions where they have a
high probability of filling to capacity in the minimum amount of time. This
means we are searching for an assignment of vehicles to regions such that the
expected time it would take to fill to capacity is minimized. This expected
time also includes the time it takes the vehicle to move from its current
location to the assigned region. We would also to ensure that the vehicle
assignment also takes into account vehicles already assigned to the regions and
candidate assignments during the search process since all vehicles in the same
region are competing for prospective requests. Lastly, we want a vehicle
assignment does not over populate the road network in certain areas (i.e.\ all
vehicles being assigned to a very fruitful region), in order to mitigate
possible self inflicted congestion which could in turn reduce the quality of
the MOD system.

\section{Approach}

\subsection{Passive Rebalancing}

We seek to find an assignment of vehicles to routes that minimizes the maximum
expected time it would take to fill any vehicle to capacity. This includes the
time it takes the vehicle to move to the origin of the route and the expected
time the vehicle would need to stay in the region to fill to capacity.  This
assignment should also be valid for vehicles that are already carrying
passengers. This means the destination of passengers needs to be taken into
consideration. Finally, in order to avoid self inflicted congestion, this
assignment should not overburden regions of the road network.  We have
formulated this problem as a mixed integer quadratically constrained problem
(MIQCP) which finds the optimal assignment of vehicles to routes and the
expected amount of time a set of vehicles would need to stay in an origin
region to all fill to capacity.

Let $V = \{1, \ldots, N_v\}$, $R = \{1, \ldots, N_r\}$, and $O = \{1, \ldots,
N_o\}$, be the set of vehicles, routes, and origins respectively. Let $A \in
\{0, 1\}^{N_v \times N_r}$ be a binary matrix such that if $A_{vr} = 1$, then
vehicle $v$ can satisfy route $r$. This can be computed using the shareability
graph.  Let $C_{vr} \in \mathbb{R}^{N_v \times N_r}$ be a cost matrix where
$C_{vr}$ represents the cost of vehicle $v$ satisfying route $r$. For our
examples we use the distance from $v$ to the origin of $r$. Let $\Lambda$ of
size $N_r$ be the set of expected request rate for all the routes. More
specifically, $\Lambda_r = \mathbb{E}(\hat{\lambda_r})$ where $\hat{\lambda_r}$
is the estimated rate of requests at the current time. Also let $\Omega$ of
size $N_\omega$ be a set such that $\Omega_\omega$ is the set of requests with
origin $\omega$ and let $K_{\omega}$ be the remaining number of vehicles we can
allow into region $\omega$. The equations below show the MIQCP formulation.

\begin{equation*}
\begin{array}{lll}
\argmin{x_{vr}, \,T_r}  & \theta &\\
% \argmin{x_{vr}, \,T_r}  & \Max{v \in V, r \in R} A_{vr} \cdot x_{vr} \cdot (T_r + C_{vr})&\\
\text{subject to}
    & x_{vr} \cdot (T_r + C_{vr}) \leq \theta,
    & \forall v, r \in V \times R\\
    & \Sum{v \in V} x_{vr} \cdot \xi_v \leq \Lambda_r \cdot T_r,
    & \forall r \in R \\
    & \Sum{r \in R} A_{vr} \cdot x_{vr} = 1,  &\forall v \in V \\
    & \Sum{v \in V} \Sum{r \in \Omega_{\omega}} x_{vr} \leq K_{\omega},  &\forall \omega \in O\\
    & x_{vr} \in \{0,1\} & \forall v, r \in V \times R\\
    & T_{r} \in [1, T_{\max}] & \forall r \in R \\
\end{array}
\end{equation*}

\subsection{Active Rebalancing}

\begin{equation*}
\begin{array}{lll}
\argmax{\bx, \, \by}  & \Sum{v \in \mathcal{V}}
    \Sum{\pi \in \Pi_v} \bx_{v\pi} \cdot
    \Sum{r \in \mathcal{R}(\pi)}
    \frac{1 - e^{-\bar{\lambda}_r(T_{\pi r})}}
    {1 + \Sum{\pi' \in \Pi} \by_{\pi'} \frac{\tau(T_{\pi r} \cap T_{\pi' r})}
    {\tau(T_{\pi r})}}& \\
\text{subject to}
    & \Sum{\pi \in \Pi_v} \bx_{v\pi} = 1, \, \forall v \in \mathcal{V} \\
    & \Sum{\pi \in \Pi} \by_{\pi} = N_v \\
    & \bx \in \{0, 1\} ^ {N_v \times N_\pi} \\
    & \by \in \{0, 1\} ^ {N_\pi} \\

\end{array}
\end{equation*}

\begin{equation*}
\begin{array}{lll}
\argmax{\bx, \, \by}  & \Sum{v \in \mathcal{V}}
    \Sum{\pi \in \Pi_v} \bx_{v\pi} \cdot
    \Sum{r \in \mathcal{R}(\pi)} \,
    \Int{T_{\pi r}}
    \frac{t \cdot \bar{\lambda}_r(t)}
    {1 + \sum\limits_{\pi' \in \Pi} \by_{\pi'} \cdot \ell(\pi'(t))} \text{d}t& \\
\text{subject to}
    & \Sum{\pi \in \Pi_v} \bx_{v\pi} = 1, \, \forall v \in \mathcal{V} \\
    & \Sum{\pi \in \Pi} \by_{\pi} = N_v \\
    & \bx \in \{0, 1\} ^ {N_v \times N_\pi} \\
    & \by \in \{0, 1\} ^ {N_\pi} \\

\end{array}
\end{equation*}

\bibliographystyle{IEEEtran}

\bibliography{bib}

\end{document}
